\documentclass{article}
\usepackage[letterpaper, portrait, margin=1in]{geometry}
\setlength\parindent{0pt}
\usepackage{natbib}

\usepackage{graphicx}
\usepackage{amsmath}

\author{Margaret J. Foster\thanks{margaret.foster@duke.edu}}
\title{Working Paper: Franchising For Terror Moderation}
\date{\today}

\begin{document}
\maketitle

The following is an exploratory study to see if the hotel case does
apply as well to the terrorism case as I think it does.


%[[NOTE IF YOU GO AHEAD WITH THIS, YOU'LL NEED TO CHANGE THE WORDS AND ADD CITATIONS!]]

\section{Research Question}

Why does franchising represent a puzzle for terror groups?
%[[ is it a point of vulnerability to reach out and connect with other
%terror groups? If so, this might be a worthwhile avenue of investigation ]]
raises attention paid to the organization, reduces potential avenues of
communications, reduces autonomy. 

Articulation of the potential drawbacks of joining a transnational
``brand,'' comes from no less a figure driving the networking as
former al-Qaeda leader Usama bin Laden.  In documents seized from his
Abottabad compound, bin Laden is recorded as rebuffing the Somali
Shabaab al-Mujahideen's attempt to join with them and clearly
describing the detrimental consequences of a formal merger:

\begin{quote}

"If the matter becomes declared and out in the open, it would have the
enemies escalate their anger and mobilize against you; this is what
happened to the brothers in Iraq or Algeria. It is true that the
enemies will find out inevitably; this matter cannot be hidden,
especially when people go around and spread this news. However, an
official declaration remains to be the master for all proof. Also,
there will be fields open to those who would like to provide rescue
assistance to Muslims in Somalia, to deny this reality which is not
based on definitive evidences. Therefore, such would minimize the
restraints on Muslims in the region of the emirate and likewise on the
emirate proper...

"I have a determined plan of action, using one of my sermons to press
the merchants in the countries of the Arabian Peninsula to support
pro-active and important developmental projects which are not
expensive; we happened to have tried these in Sudan. Therefore, by not
having the mujahidin openly allied with al-Qa'ida, it would strengthen
those merchants who are willing to help the brothers in Somalia, and
would keep people with the mujahidin." (CTC,SOCOM-2012-0000004-HT)
\end{quote}

Moreover, some of the benefits of linkage may be ambiguous: additional
resources and access to tactical innovations, yes. Yet, short of the
deployment of trainers- which, occasionally happens even without an
official merger for geostrategicially important groups, 
%[[Find UN                                %document of Hezbollah training
   %Shabaab fighters - used in Badreddine
                                %report]]
- tactical transmission can happen through other, less costly,
channels- such as Google.
 
- For the "base" group, spawning a franchise:
\begin{itemize}
\item raises profile (a double-edged sword, as
presented above, but also increases the managerial headaches, and
increases the necessity of a paper trail... which can be
catastrophic.
 \item Forces a group to have to network together, under heavy
surveillance,
\item ties base group's reputation to that of subgroups that
may or may not be completely under their control.. %[[Here: UBL on
                                %needing to reframe AQ away from
\end{itemize}
 
\section{Framework}

The terrorism "industry" increasingly features large, transnational, "firms" targeting
global audiences with their advertising and media campaigns.\footnote{For
example, Phillips (2013) finds that al-Qaeda- a relatively recent terror group- is the entity with the
most alliances across his study of terror group alliances.}

Yet, despite affiliations with transnational networks, individual
terrorist groups operate at local levels where they must compete with
potential rivals for support. The relative positions of subcomponents
of large, networked, terror groups can differ significantly across
geographic areas of operations, with al-Qaeda affiliated groups
preponderant in venues such as North Africa, Somalia, and Yemen; a
major, but not exclusive player, of Sunni-affiliated jihadi insurgencies
in the Levant; and one element among many in the Indian subcontinent.    

I adapt model developed by economists to identify the advantages of
hotel branding, and particularly the framework proposed by Nathaniel
Willson to explain the lack of spatial preemption in the hotel
industry \cite{wilson2011branding}.  

\section{Looking at firm background}

This local competition of a global movement poses important questions
for researchers interested in the determinants of terror group
successes and how they compete for local community support. 

Should researchers focus on the global element, or the local? Moreover this
adds to recent work on terror group cooperation ..

An emergent research agenda is ongoing in the political science
literature. There has been some preliminary work adapting models of
firm behavior to al-Qaeda, for example, \cite{zelinsky2009research},
but the area is still underdeveloped.
.%[[lack of cases?]]

A potentially useful analytic tool is to appeal to previous economic
and management work done on competition of large firms, particularly
large firms that establish overarching brand identities with
locally-overseen individual branches. One specific point of similarity is between transnational jihadi networks and
the hotel industry: as in the hotel industry, the international terror
"industry" is experiencing a new phenomenon in which a relatively small
number of large "firms" (al-Qaeda, ISIS, ...) cultivate
international reputations. 


                               %harming civilians, blasting TTP]]

\section{What a brand-centric approach buys}

Adopting brand-centric approaches helps shed light on the benefits
that might drive groups to form such networks. 

One approach to explaining the multiplicity of hotel chains in local
markets relies on heterogeneity of consumer impressions of national
firms, \cite{Wilson2011branding}Shaked and Sutton, 1987, Sutton,
2007]]. For hotel brands, which compete in the open market for
survival, ``consumer's horizontally-differentiated preferences are
believed to be of large economic significance'' in which it is
believed that ``...consumers may not exactly know why they prefer
certain firms’ hotels, their choices indicate they will pay a
significant premium to stay in a hotel affiliated with their preferred
firm'' \cite{wilson2011branding}.

What connects terror recruitment with the hotel industry specifically-
instead of other industries denoted by franchising, such as fast food
and the gasoline industry, is that both hotel and terror recruitment
share a tendency not to have repeated instances of choice.

 an effect that is intuitively likely to also be at play in the
receptiveness of global terror "brands."

Why is the hotel industry a useful approach to studying global jihadi
movements? There are some structural similarities: both hotels and
global militant groups should suffer from acute race-to-the-bottom
effects. Hotels cater to generally non-repeat clients, which
theoretically makes it difficult for hotels to credibly signal
quality. Since their client base does not generally have the
opportunity to punish if the promised quality is missing, hotels
should have a strong incentive to engage in a race to the bottom. One
way that hotels can escape this trap is to be a part of a national
chain that customers are likely to be able to chose from again.  [[So,
why are B\&B's usually pretty nice?]] ."The ability to be collectively
punished allows multi-hotel firms to credibly commit to higher levels
of quality. Moreover, vertical differentiation in the quality of
services provided weakens competition among different quality
segments: similar to how terror groups are likely to only be competing
for a small segment of the base of potential supporters from within
the entire population. They're effectively like a specific strata
within the hotel industry: unlikely to gather patronage from those
looking for other amenities or for services at a different cost level.
 

The underlying patterns appear to be equally applicable to the
challenges faced by a terror group attempting to establish a franchise
model. In this case, heterogeneous "brand" preferences of potential supporters
reduces the ability of terror groups to engage in spatial preemption,
moving into one area to forestall the development of rival
groups. Thus, the attempt of al-Qaeda in Iraq to establish an
outgrowth in Syria did not result in the group crowding out all other
potential terror groups. Indeed, al-Qaeda in Iraq's attempt to
capitalize on the Syrian Civil war didn't even spatially preempt the
group competing with itself, as their vanguard, the al-Nusra Front,
acrimoniously split from al-Qaeda in Iraq.

 Instead, terror organizations that operate on a franchise model tend
 to wait for established organizations to grow an consolidate ``market
 share'' before moving in and co-opting the existing base.

Similarly to how the economics of hotel chains makes it profitable for
hotels to concentrate on those travelers who are most strongly
committed to their offerings, terror groups have an incentive to for
terror ``brands'' to concentrate on those potential supporters who
most strongly prefer their platform. Factional or ethnic differences
may splice the potential market for terror ``brands'': al-Qaeda
affiliated groups are unlikely to ever capture the support of
potential Hezbollah supporters, except perhaps those who are far
removed from the battlefield and who offer week support for any group
acting against a common enemy, such as Israel. However, this weak base
of support is not likely to be worth the effort and danger of reaching
out. 

Moreover, in a crowded militant group marketplace- or a marketplace
with strong government capacity- it may be actively
dangerous for terror groups to focus on all but their own
partisans. Attempting to reach out provides information about a group,
which could be used for a government crackdown or  agression from
rivals. Thus, it isn't useful for, say, al-Qaeda recruiters to attempt to recruit
foreign fighters in Leicester Square, or even moderate mosques.
Instead, previous recruitment efforts were concentrated on specific
sites affiliated with Sunni radicalization, such as the North
London Central Mosque and the Taiba Mosque in Hamburg, Germany.

The underlying patterns appear to be equally applicable to the
challenges faced by a terror group attempting to establish a franchise
model. In this case, heterogeneous "brand" preferences of potential supporters
reduces the ability of terror groups to engage in spatial preemption,
moving into one area to forestall the development of rival
groups. Thus, the attempt of al-Qaeda in Iraq to establish an
outgrowth in Syria did not result in the group crowding out all other
potential terror groups. Indeed, al-Qaeda in Iraq's attempt to
capitalize on the Syrian Civil war didn't even spatially preempt the
group competing with itself, as their vanguard, the al-Nusra Front,
acrimoniously split from al-Qaeda in Iraq.

 Instead, terror organizations that operate on a franchise model tend
 to wait for established organizations to grow an consolidate ``market
 share'' before moving in and co-opting the existing base.

Similarly to how the economics of hotel chains makes it profitable for
hotels to concentrate on those travelers who are most strongly
committed to their offerings, terror groups have an incentive to for
terror ``brands'' to concentrate on those potential supporters who
most strongly prefer their platform. Factional or ethnic differences
may splice the potential market for terror ``brands'': al-Qaeda
affiliated groups are unlikely to ever capture the support of
potential Hezbollah supporters, except perhaps those who are far
removed from the battlefield and who offer week support for any group
acting against a common enemy, such as Israel. However, this weak base
of support is not likely to be worth the effort and danger of reaching
out. 

Moreover, in a crowded militant group marketplace- or a marketplace
with strong government capacity- it may be actively
dangerous for terror groups to focus on all but their own
partisans. Attempting to reach out provides information about a group,
which could be used for a government crackdown or  agression from
rivals. Thus, it isn't useful for, say, al-Qaeda recruiters to attempt to recruit
foreign fighters in Leicester Square, or even moderate mosques.
Instead, previous recruitment efforts were concentrated on specific
sites affiliated with Sunni radicalization, such as the North
London Central Mosque and the Taiba Mosque in Hamburg, Germany.


The underlying patterns appear to be equally applicable to the
challenges faced by a terror group attempting to establish a franchise
model. In this case, heterogeneous "brand" preferences of potential supporters
reduces the ability of terror groups to engage in spatial preemption,
moving into one area to forestall the development of rival
groups. Thus, the attempt of al-Qaeda in Iraq to establish an
outgrowth in Syria did not result in the group crowding out all other
potential terror groups. Indeed, al-Qaeda in Iraq's attempt to
capitalize on the Syrian Civil war didn't even spatially preempt the
group competing with itself, as their vanguard, the al-Nusra Front,
acrimoniously split from al-Qaeda in Iraq.

 Instead, terror organizations that operate on a franchise model tend
 to wait for established organizations to grow an consolidate ``market
 share'' before moving in and co-opting the existing base.

Similarly to how the economics of hotel chains makes it profitable for
hotels to concentrate on those travelers who are most strongly
committed to their offerings, terror groups have an incentive to for
terror ``brands'' to concentrate on those potential supporters who
most strongly prefer their platform. Factional or ethnic differences
may splice the potential market for terror ``brands'': al-Qaeda
affiliated groups are unlikely to ever capture the support of
potential Hezbollah supporters, except perhaps those who are far
removed from the battlefield and who offer week support for any group
acting against a common enemy, such as Israel. However, this weak base
of support is not likely to be worth the effort and danger of reaching
out. 

Moreover, in a crowded militant group marketplace- or a marketplace
with strong government capacity- it may be actively
dangerous for terror groups to focus on all but their own
partisans. Attempting to reach out provides information about a group,
which could be used for a government crackdown or  agression from
rivals. Thus, it isn't useful for, say, al-Qaeda recruiters to attempt to recruit
foreign fighters in Leicester Square, or even moderate mosques.
Instead, previous recruitment efforts were concentrated on specific
sites affiliated with Sunni radicalization, such as the North
London Central Mosque and the Taiba Mosque in Hamburg, Germany.

The underlying patterns appear to be equally applicable to the
challenges faced by a terror group attempting to establish a franchise
model. In this case, heterogeneous "brand" preferences of potential supporters
reduces the ability of terror groups to engage in spatial preemption,
moving into one area to forestall the development of rival
groups. Thus, the attempt of al-Qaeda in Iraq to establish an
outgrowth in Syria did not result in the group crowding out all other
potential terror groups. Indeed, al-Qaeda in Iraq's attempt to
capitalize on the Syrian Civil war didn't even spatially preempt the
group competing with itself, as their vanguard, the al-Nusra Front,
acrimoniously split from al-Qaeda in Iraq.

 Instead, terror organizations that operate on a franchise model tend
 to wait for established organizations to grow an consolidate ``market
 share'' before moving in and co-opting the existing base.

Similarly to how the economics of hotel chains makes it profitable for
hotels to concentrate on those travelers who are most strongly
committed to their offerings, terror groups have an incentive to for
terror ``brands'' to concentrate on those potential supporters who
most strongly prefer their platform. Factional or ethnic differences
may splice the potential market for terror ``brands'': al-Qaeda
affiliated groups are unlikely to ever capture the support of
potential Hezbollah supporters, except perhaps those who are far
removed from the battlefield and who offer week support for any group
acting against a common enemy, such as Israel. However, this weak base
of support is not likely to be worth the effort and danger of reaching
out. 

Moreover, in a crowded militant group marketplace- or a marketplace
with strong government capacity- it may be actively
dangerous for terror groups to focus on all but their own
partisans. Attempting to reach out provides information about a group,
which could be used for a government crackdown or  agression from
rivals. Thus, it isn't useful for, say, al-Qaeda recruiters to attempt to recruit
foreign fighters in Leicester Square, or even moderate mosques.
Instead, previous recruitment efforts were concentrated on specific
sites affiliated with Sunni radicalization, such as the North
London Central Mosque and the Taiba Mosque in Hamburg, Germany.


\section{Continuation}

[link to terrorism: does connection with a transnational group exert a
pressure on an organization to engage in relative moderation in their
choices of targets? Their rhetoric is pretty clear that this kind of
networking is supposed to extert at least a somewhat moderating effect
on target selection- the organizations are brutal, but tend to claim
relatively low "cost"/ high effect attacks.  

Just as the bulk of corporate hotels are of "high quality," the bulk of independent hotels are of "low quality"
 The marketing strategies of large terror network are large and sophisticated, and feature prominently in the group strategy 

This framework also provides an additional tool to analyze the attempts of AQ to empower and mobilize lone wolves: much as hotel chains create multiple tiers of outlets to capture travelers with different budgetary constraints, the al-Qaeda network is vertically differentiating. They've gone to great lengths to create a "budget" option that is co-branded with al-Qaeda... but which is a much more accessible option. The effect is similar to how the Marriott brand has similarly-named mid and high tier options (JW Marriott/ Marriott/ Marriott Courtyards). For those potential recruits unable to access the costly option- emigrating and fighting with one of the AQ affiliates where they can gain battleground training, the group has spun-off a movement. Here differentiate movement from organized groups- Brown's APSA 2014 discussion of ELF and Anon from the APSA panel will work .  Both the ELF and Anonymous are operating in the more "downmarket" sector where they are attempting to empower individual action and  use atomized individual actions to build their culture into a movement.  If you are part of a culture that wants to be a movement, claiming responsibility helps ignite movement and create a template for future. (Brown, APSA 2014). 
 
Underscoring the similarities of how "downmarket" terror movements operate- the ELF operating procedures are astonishingly similar to the instructions for Lone Wolves provided by AQ:

-ELF coordinates through claims and PGP email... basically the same as
the Inspire mobilization campaigns. In the case of the ELF, the press
office ensures compliance with the ELF tenants by threatening to not
promote operations that don't adhere to ELF agenda. (Brown, APSA
2014).  AQAP/ AQ Central has made it a bit more difficult for
themselves  to exert that sort of leverage over potential
attackers. Instead, they've publicized specific suggested targets,
and, by being at arms-length from those they've enabled, the group


\section{Implications and testable empirics}

The ability to collectively punish helps ameliorate races to the ultra-violent fringe, which one might expect to see from
violent militant groups. So, franchising represents an attempt to
opt-out of race to the bottom tactics. 
% [[are there any vote-placmenet models about how groups race to the
% fringe if voters can opt-out. ]]

Examples:
\begin{enumerate}
 
\item AQAP hospital bombing apology
\item AQ/ ISI discussion about targeting civilians and large-scale violence     
\item Jihadi forum debates about supporting Boko Haram attacks against civilians
     
\end{enumerate}

%[[Other groups that have broadcast ahead
                           %of time require strength. Joining a
                           %franchise might be a less-costly use of
                           %this. The IRA also claimed attacks before
                           %they carried them out generally understood as a credit-claiming
  %tool to avoid making it possible for others to try to benefit from
  %their operations, it might also fit in this framework- I'm not sure
  %that this fits. ]]

\section{Expectations}

\begin{enumerate}

\item When plausible deniability, more high-casualtiy brutal attacks
  will go officially unclaimed.  Elements that enhance plausible
  deniability include a diffuse backing- so that the supporters of the
  brand don't have specific on the ground knowledge and must relie on
  second-hand information.
\item Strongly networked groups should violate Potter and Abraham's expectation
that increases in civilian victimization follow after leadership change-overs
\item franchising will not tend to create homogenous local terrorism ``markets'' \\

\end{enumerate}

On Item three above: although the literature might suggest that firms enjoy a first-mover advantage and
can preserve market power by "spatially preempting" other firms, local
hotel markets are rarely as homogeneous as the spatial preemption
argument would suggest. This effect is mirrored on the terrorism side,
where, despite the theoretical advantages of participation in a
network, al-Qaeda and ISIS-linked groups rarely crowd out the local
terror market.\\

the pressure to avoid the race to the bottom and promotion of
  relative moderation may also be why AQ-linked groups tend to signal
  their attacks beforehand 


\bibliographystyle{plainnat}
\bibliography{Franchising.bib}
\end{document}
