\documentclass{article}[12pt]
\usepackage{setspace}
\usepackage{listings}
\lstset{language=R}   
\doublespacing
\usepackage[letterpaper, portrait, margin=1in]{geometry}
\setlength\parindent{0pt}
%\setlength{\parindent}{4em}
\usepackage{multirow}
\usepackage{graphicx}
\usepackage{amsmath}
\usepackage[sort,round]{natbib}
\usepackage{fancyhdr}
\pagestyle{fancy}

\begin{document}
1/16/16 Initialization
this is a document to provide 1-2 paragraph writeups of papers that I have read in classes. Short-term goal is for quals prep.
the document gets backed up to \\
https://github.com/margaretfoster/PaperSummaries, so that I can have Beeminder track my quals prep 
while I develop good habits.

Summary outlines:

Keywords provide a shorthand for broad type of paper (theory development, disaggregation, test theory, introduce data., literature review....); empirical approach (formal model, quant, computational)

Structure and keywords:

Citation

\underline{Keywords:}

\underline{Research question and research area}\\
\underline{Summary of argument and mechanism}\\
\underline{Data \& methods}\\
\underline{Findings}\\
\underline{Notes on related papers}\\



\section{International Relations}
(Paper summaries for Democratic Durability project and for traditional IR reading)


\textbf{Gunitsky, Seva. "From Shocks to Waves: Hegemonic Transitions and Democratization in the Twentieth Century." International Organization 68.03 (2014): 561-597.}

Gunitsky proposes a theory of “wave” of institutional change in state regime types in order to answer the question of why democratization of states has occurred in spurts rather than linearly. The mechanism is that that changes in the relative power of hegemons— particularly sudden changes or shocks— creates “windows” in which it is relatively less costly for a hegemon to export their own regime type, both through direct imposition and through influence and the design of international institutions. [[Observe that there is an assumption that hegemons want to export their own regime types.]]
\underline{Empirical test}, based on country-year from 1900- 2000, with national level of democracy as the dependent variable (measured through Polity IV and a Boix, Miller, and Roasto binary measure); shifts in power conducted through changes in CINC score. The supplementary materials indicate that the models he used are OLS for the version with a Polity DV and logit for the binary DV.  A followup model used fixed-effects specification for each of three hegemons.
\underline{Concerns:} Assumption that post CW is driven by US strengthening and so is a democracy-promotion period, but the cinc score that he uses actually shows the US remaining the same and China rising relative to the Soviet Union. [[Unclear what type of regime China would be exporting.]] Concern about whether OLS/Logit is appropriate modeling strategy.


\section{Social Networks / Behavioral}
(Paper summaries for 2016 Poli Sci Social Networks Class)

\textbf{Ryan, John Barry. ``Social networks as a shortcut to correct voting.'' American Journal of Political Science 55.4 (2011): 753-766. APA}\\
Ryan presents the results of an experiment about how people use information from their social network to determine which of several candidates is likely to best represent their interests. The motivation is a gap between the predictions of formal theory, which predicts that listeners should judge the credibility of messengers according to their individual merits, and the \textit{autoregressive influence theory} that proposes that speakers are influential to the degree that their messages conform to the other messages that the voter is receives (or accepts?). \underline{Implied Mechanism:} The paper tests the salience mechanism, in which messages from other actors, with potentially divergent preferences, influences how actors weigh their decisions.

The experiment is set up so that individuals decide on one of two computer-generated candidates to vote for. There are three types of voters: 1) control group has no contacts with other players while the treatment group exchanges messages; 2) partisan voters get extra benefits if their candidate wins, independent voters are indifferent; 3) informed voters get extra information about the non-partisan benefits of each candidate. The \textit{results} are consistent with the predictions of the autoregressive influence theory: uninformed voters benefited from having social information, informed subjects were harmed unless they were primarily contacted by people with the same preferences that they had. People tended to vote as their social networks voted. The magnitude of the effects seems very small.

\textbf{Leighley, Jan E. 1990. “Social Interaction and Contextual Influences On Political Participation.” American Politics Research 18(4): 459–475.}\\

\underline{Keywords:} social networks; voting; empirical test of theory

\underline{Research question and research area:} Leighley tests then-current theories about how social networks affect political participation. She network size, politicization, and within-network political conflict as predictors of four indicator variables: voting, campaigning, contacting, and 
\underline{Summary of argument and mechanism:} Leighley is carrying out an empirical test of arguments that social networks influence activity. Both she and the theories that she cites seem to have a strong assumption of community homophily.
\underline{Data \& methods:} Data from the 1976 American National Election Study (ANES) survey, in which respondents were asked to list the people with whom they discussed the ``most important national problem.''  I think that the social network IDVs are brought in via a count of discussants and indicator variables for conflict and politicization. Empirical test is a logit model of 724 individuals.
\underline{Findings};
\underline{Notes on related papers or followup:} The conflict measure would be interesting to look at in a multiparty system, with more fluid voting patterns. Would also be interesting to look at network-based diffusion of strategic voting. 

\textbf{McAdam, Doug, and Ronnelle Paulsen. 1993. “Specifying the Relationship between Social Ties and Activism.” American Journal Of Sociology 99(3): 640–67.}

\underline{Keywords:} networks; theory-testing

\underline{Research question and area:} McAdam and Paulsen connect to the literature of mobilization into high-risk political organizations. They primarily connect with the sociology literature, and, for example, don't make reference to free-rider problems that preoccupy the civil wars perspective on this question.\\
\underline{Summary of argument and mechanism:} McAdam and Paulsen are testing theories about how interpersonal ties and membership in organizations influence identity-based mobilization. The theory holds that social networks reinforce identity-based recruitment appeals by affirming that participation would support salient identity.
\underline{Data \& methods} Social network data generated from 1068 applications for participation in the 1964 Mississippi Freedom Summer project, coded networks based on surveys from participants, rejections, and no-shows and 556 followup surveys. Network-related independent variables operationalized as binary variables coding presence of listed organization or source of support.  Dependent variable was binary for participation in freedom summer; logit (?) statistical model.
\underline{Findings:} Personal ties are most important as sources of influence (vs conduits for information); suggestion that best networks for mobilization is weak ties between lots of tightly linked local organizations. Organizational support more important than individual ties for encouraging participation in a collective action.
\underline{Notes on related papers:} Lots of applicability here to mobilization for violent extremist organizations. 

\textbf{Mutz, Diana C. 2002. “The Consequences of Cross-Cutting Networks for Political Participation.” American Journal of Political Science 46(4): 838–55.}

\underline{Keywords:} networks; theory testing

\underline{Research question and research area:} Does homogeneity of political belief in networks influence political participation?
\underline{Summary of argument and mechanism:} Non-homogeneous networks introduce uncertainty by transmitting contradictory information thus making people take longer to decide who to vote for; non-homogenous networks depress willingness to publicly share a position because can not make everyone in the network happy.
\underline{Data \& methods:} Used two 1996 surveys about political participation and political interaction within social networks. Logit model with network information included as binary and count variables.
\underline{Findings:} Results most strongly suggest that cross-cutting networks depress willingness to be demonstrative, rather than that the influence makes people confused about what they want.
\underline{Notes on related papers:} I feel like the conflict-avoidance mechanism here would be a natural fit to update to the 21st century via questionnaires about people's activities on social networking sites. (\textit{e.g.}: in the past x months, have you avoided sharing content that promotes a political or social view because of concern about how people will respond?)

\textbf{McClurg, Scott D. 2003. “Social Networks and Political Participation: The Role of Social Interaction in Explaining Political Participation.” Political Research Quarterly 56(4): 449–64.}

\underline{Keywords:}

\underline{Research question and research area:} Seems to speak most closely to the social capital literature and social determinants of political participation. 
\underline{Summary of argument and mechanism:} Mechanism is explicitly about information-sharing functions of social networks.
\underline{Data \& methods:} Survey data from 1984 Presidential election gathered in South Bend, Indiana. Unit of analysis is the discussion dyad. Dependent variable is a composite count of political activities beyond voting. Model is negative binomial regression.
\underline{Findings:} 
\underline{Notes on related papers:}
combining this article and the  Mutz findings and adding to the context  of social networks is interesting:
then we have a  clear vector of information (both  passive via FB  events) and active,  via people sharing  their views and  things like articles,  BUT also because some online social networks tend to be maximalist (especially Facebook) they also foster the  development of the type of cross- cutting networks that Mutz finds tends to make people cagy about political activism. I bet that beyond presidential election, social campaigns like support for Black Lives Matter would be a good test case of information sharing through networks (having partisans in online networks might increase access to POV and stories.

\textbf{Nickerson, David W. 2008. “Is Voting Contagious? Evidence from Two Field Experiments.”
American Political Science Review 102(1): 49–57.}

\underline{Keywords:} voting behavior, networks
\underline{Research question and research area} This article is looking at testing one of the mechanisms behind why spouses exhibit similar behavior: whether they converge together or pre-select for similarity.\\
\underline{Summary of argument and mechanism} Network mechanism being tested is information exchange through strong ties. Update: he does not talk about mechanism. He's focused on trying to show causality, not mechanism.\\
\underline{Data \& methods} Experiments in Denver, Co and Minneapolis, MN for 2002 congressional primaries.\\
\underline{Findings} Canvassing resulted in an 8\% increase in propensity to vote among contacted people, and a 5.5 (CO) and 6.5\% (MN) increase among cohabitants of the person contacted. Wording is vague, bit it looks like significance didn't meet .05 threshold but did meet .1. Tried to be careful, but could not completely rule out some selection mechanism.\\
\underline{Notes on related papers} They're using ``spouse'' and ``roommate'' interchangeably in their mechanism description. This implies that the mechanism is supposed to be just information exchange, particularly in an environment of strong ties, vs pressures driving spouses to similarity.   \\


\textbf{Lazer, David, Katya Ognyanova, William Minozzi, and Michael Neblo. “The social control of
political participation: Conflict and contagion as processes (de)mobilizing voting.” Working Paper.}

\underline{Keywords:} mechanism testing

\underline{Research question and research area} The paper tries to test the contagion and conflict mechanisms of network effects on voting behavior\\
\underline{Summary of argument and mechanism} No proposed mechanism, trying to experimentally test debate in literature about whether conflict or contagion effects are at work in networks of undergrads\\
\underline{Data \& methods} Data is from 14 dorms in election years from 2008-2012\\
\underline{Findings} Findings support contagion, although I suspect that their survey question doesn't quite parse out the question of whether people frequently talk about politics with those whom they disagree. \\

\underline{Notes on related papers}\\


\textbf{Franzese, R. and J. Hays. 2008. “Interdependence in Comparative Politics: Substance, Theory, Empirics, Substance.” Comparative Political Studies 41: 742–80.}

\underline{Keywords:} spatial processes; diffusion; review

\underline{Research question and research area}\\
\underline{Summary of argument and mechanism} Cite Simmons et al (2005) on four mechanisms of international diffusion: direct/indirect coercion; competition via economic pressures; learning from others' actions; emulation; (they add) migration where some components directly enter others. Strategic interdependence is when some units' actions affect marginal utilities of actions for another.  Free-riding occurs when policies are strategic substitutes.\\
\underline{Data \& methods} Nice overview on pages 753-654 of the models implied by different views of how interconnectedness influences outcome.  \\
\underline{Findings} Case study is from international diffusion of tax policies.\\
\underline{Notes on related papers}\\

\section{Terrorism}

\textbf{de Mesquita, Ethan Bueno. "The political economy of terrorism: A selective overview of recent work." The Political Economist 10.1 (2008): 1-12.
APA} (Week one)

Key: literature review.
 
This is a literature review essay about the then-current strand of the literature of academic study of terrorism, with a particular focus on the outcomes of formal models.  There is not a theory that is being tested, so there isn't data to assess. Note that, as with all EBdM papers about terrorism, he presents the results of his own research as if the conclusions are settled questions in the field.

\textbf{Aksoy, Deniz and David B. Carter. 2014. “Electoral Institutions and the Emergence of Terrorist Groups.” British Journal of Political Science 44(1): 181-204.}

Key: disaggregation; quant.

Research question and area: Aksoy and Carter build on the literature about how domestic institutions-- and particularly democratic institutions-- influence the emergence of terror groups. They are in the `` do democracies experience more terrorism'' research agenda. A\&C look at how different types of terror groups (``within-system'' who have goals that can be fulfilled within existing system and ``anti-system'' whose goals require complete overthrow of the existing political system. Irredentist and separatist groups are called anti-system.) are differently influenced by democratic institutions, and argue that previous research hasn't found ``robust'' evidence of a connection between electoral rules and terrorism because it combines all groups together. Thus, this is a \underline{disaggregation} type paper.
Articulated/assumed definition of ``terrorism'': Terrorism is a political organization trying to ``achieve their goals by using violence against civilians''
Summary of argument and mechanism: Democratic institutions should reduce operation of within-system groups because they provide a way for the goals of the organization to be met; should not have an impact on anti-system groups.
Data \& methods: Look only at domestic groups; data is from Jones and Libick (2006), which has 648 terror grops from 1968-2006, and TWEED, which has 286 groups from 150-2004. Three main DVs: emergence of any group, emergence of anti-system group, emergence of within-system group. Unit of analysis is country-year. Primary IDVs: indicator variable for democracy, domestic institutions from IAEP and Democratic Electoral Systems Around the World Project, log of district magnitude, and various controls. Model is logit.
Findings: permissive electoral systems associated with reduced risk of within-system group emergence but have no impact on anti-system groups. 
Notes on related papers:

\textbf{Blomberg, S. Brock, Gregory D. Hess and Akila Weerapana. 2004. “Economic Conditions and Terrorism.” European Journal of Political Economy 20(2):463-478.}

\underline{Keywords:} quant; computational; theory testing.

\underline{Research question and area:} Blomberg et al. contribute to the literature on how economic conditions influence terrorist activities.\\
Terrorism definition: differentiate between ``rebellion'' which happens when a group that wants to change the status quo overthrows the government and takes power and ``terrorist attacks'' when dissident groups use ``terror activities'' to ``increase voice in economy'' but are not able to overthrow government. Note the almost tautological definition and the build-in assumption that terror attacks are intended to express a perspective on economy. Terror attacks conceptualized as ``low-intensity attacks'' designed to signal discontent with status quo. [This definition would include the far-left / far-right vandalism.]
\underline{Summary of argument and mechanism:} Groups choose between rebellion and terror attacks according to whether the country can give in to the dissidents. Strong institutions attracts terrorism; weak ones attract rebellion. [Observe that this outcome is built into their definition: weaker states should disproportionately attract attacks that threaten the state institutions because the institutions are more vulnerable.] They have a strong underlying assumption that attacks are intended to signal discontent. Assumption that attacks are either signals of anger or earnest attempts to overthrow government.
\underline{Data \& methods:} Dataset of 130 countries from 1968-1991, country-year unit of analysis. Data from ITERATE, which is transnational attacks and thus doesn't really match their theory, with economic indicators. Build a model of economic state change and terrorism/no terrorism through Markov processes. Do not appear to have any measure of the rebellion actions of their theory and look at conditional probabilities of state change. [Methodological question: are Markov chains at all appropriate for strategic interactions? In general, they are only applicable for linear systems.]
\underline{Findings:} In democratic, high income, countries, recessions increase probabilities of terror attacks. 
Notes on related papers:


\textbf{Chenoweth, Erica. 2013. “Terrorism and Democracy.” Annual Review of Political Science 16: 355–378.}

\underline{Keywords:} literature review.
\underline{Research question and research area:} Chenoweth provides an overview of research on the question f whether democracies disproportionately attract terrorist violence. \\
\underline{Operational definition of terrorism:} Here Chenoweth works from the basis that terrorsim is the deliberate use or threat of force against noncombatants by a non-state actor in pursuit of a political goals.
\underline{Summary of argument and mechanism:} Surveys underlying arguments about why democracies might, or might not, attract terrorism that are found in the articles that she synthesizes. 
\underline{Data \& methods:} Varies based on paper surveyed, but in general the studies all based on observed attacks and the GTD features heavily in both the motivating introduction and in the studies that she discusses. 
\underline{Findings:} She synthesizes existing studies to argue that terrorism is not a strategy of desperation, in particular that there is a non-linear relationship, with more violence in anocracies and partial democracies. As well, the patterns of the data change after 2001, with terrorism shifting from democracies to partial democracies and authoritarian regimes
\underline{Notes on related papers:} As a general note- this is probably a major source, and likely even a template, for any question about democracies and terrorism.

\textbf{Drakos, Kostas and Andreas Gofas. 2006. “The Devil You Know but are Afraid to Face:
Underreporting Bias and its Distorting Effects on the Study of Terrorism.” Journal of Conflict Resolution 50(5):714-735}

\underline{Keywords:} Terrorism; methodology

\underline{Research question and research area:} This is primarily a methodological paper- Drakos and Gofas are concerned with underreporting bias over terror attacks, leading to the situation in which the count of attacks that we see in news is lower than full sample of terrorist attacks and systematically underrepresents attacks in autocracies without a free press. This paper also speaks to the democracy and terrorism literature, which is where they primarily orient their literature review.
\underline{Construction of terrorism:} Terrorism is violence, or threat of violence calculated to create an atmosphere of fear and alarm. [Observe that this allows for states to perpetuate terrorism, which might be particularly systematically underreported.] 
\underline{Summary of argument and mechanism:} Argue that bias depends on degree of press freedom in country, so less freedom means underpresentation in aggregate databases like the GTD and findings that democracies attract more terrorism. Argue that cannot just put in a control for press freedom in part because it is a component of Polity.
\underline{Data \& methods:} Use concept of ``thinning'' in which count-data is generated as the true number of events weighted by some country-specific probability of inclusion. Assume that policy is ``fundamental determinant'' of underreporting bias because of inclusion of press freedom. Data sample is country-year unit for 153 countries from 1985-1998. Data from MIPT and Rand. Determine underreporting by looking at differences between two linear models (pg 732 of article).
\underline{Findings:} Find evidence that they claim is strongly indicative of underreporting bias in many countries, but they avoid making country-specific observations.
\underline{Notes on related papers:} Wonder if one could use Bayesian techniques to extend this approach


\textbf{Krueger, Alan B. and David Laitin. 2008. “Kto Kogo?: A Cross-Country Study of the Origins and Targets of Terrorism.” In Terrorism and Economic Development, ed. Philip Keefer and Nor- man Loayza. New York: Cambridge University Press.}

\underline{Keywords:} quant; theory testing

\underline{Research question and research area}\\  Krueger and Laitin write as part of the debate about whether macroeconomic conditions (\textit{e.g.} poverty) motivate people to commit terrorism. 
\underline{Construction of terrorism:} US State department definition: premeditated, politically motivated violence perpetrated against non-combattant targets by subnational groups or clandestine agents, usually intended to influence a target audience. [Observe: no implication of lethality in this definition, and the definition is squishy about the degree of messaging intended.  Unclear, for example, how this classification would handle Hezbollah in Syria.]
\underline{Summary of argument and mechanism}\\ There isn't really a unique theoretical argument here- Kruger and Laitin are trying to test existing competing predictions about whether poverty facilitates terror recruitment.
\underline{Data \& methods}\\ Used State department data on 781 ``significant'' international terror attacks for the first database and 236 suicide attacks from Pape and International Policy Institute for Counter-Terrorism (ICT) for the second databases, with independent variables about the perpetrator (if country of origin and target are the same, suicide attack, multiple attacks) and environment (GDP, terrain, religious affiliation, literacy). Unit of analysis is terrorist attack. Note that any victims that were from a country other than the one that the attack happened in registered as an international attack. Modeling is done via a cross-tab (for differences in sender-receiver) and a negative binomial regression model.
\underline{Findings}\\ Perpetrators of international terrorism tend to hit targets that are closer to the attacker's home base; countries with high civil liberties are less likely to be countries of origin while low income and low-growth countries are more likely to generate attackers

%\underline{Possible extensions} Would be really interesting to take this framework-- in particular origin of attacker less likely the greater the domestic civil liberties are--- and apply it to foreign fighter flows. The mechanism that one might use to assume that the findings might be different could be that more civil liberties means more circulation of ideological propaganda, which triggers more recruitment. 

\underline{Notes on related papers}\\ (From in-class summary) They find that the countries of lower GDP are more likely to experience terror activity, which gets close to being the opposite trend argued in the democracy-attracts-terrorism literature.

PE of Terrorism week 5:\\

\textbf{Abadie, Alberto and Javier Gardeazabal. 2003. “The Economic Costs of Conflict: A Case
Study of the Basque Country.” American Economic Review 93(1):113-132.}

\underline{Keywords:} quant; methodology

\underline{Research question and research area} \\ Attempt to quantify the economic costs of terrorism by comparing the actual GDP of Basque Country with a simulated counter-factual GDP of the same region without attacks. 
\underline{Summary of argument and mechanism}\\  This is pretty a-theoretical. They take as a given that ETA terror activity did influence the Basque County's GDP relative to the rest of Spain and then try to quantify what the difference was.
\underline{Data \& methods} Use data from other Spanish provinces that looks like 1960-era Basque to generate a weighted composite counterfactual. Weights selected so that the counterfactual ``Basque'' (going to call ``Basque-C'' from now on) was most similar to real-Basque. Evolve the provinces differently after 1975, when they say that ETA terrorism becomes ``a large-scale phenomenon'' Also use the 1998-199 truce as a experiment for how investment and economic activity changes after truce.\\
\underline{Findings} GDP in the real-Basque country declined about 10\% relative to the ``Basque-C'' control.  The real benefit from this paper would be if you want a template for other such simulation projects.\\
\underline{Notes on related papers} Interesting that we haven't really seen a ton of papers with a close focus on ETA. Maybe a gap in the literature. \\


\textbf{Abrahms, Max. 2012. The Political Effectiveness of Terrorism Revisited. Comparative Political Studies 45(3): 366-393.}


\underline{Keywords:}

\underline{Research question and research area} Does terrorist activity generate concessions for the violent group? He is writing as part of the ``does terrorism work'' debate. \\
\underline{Summary of argument and mechanism} Terrorism is unlikely to work politically because targeting civilians makes leaders unwilling, and politically unable, to make concessions. But, violence is an imperfect form of communication because target countries tend to assume that the group's demands are too extreme for the domestic political context.
\underline{Conceptualization of terrorism} Inclusion criteria are: designated by US, have killed at least one person from the targeted country for stated reason of announcing concessions; differentiates guerrilla from terror by whether the majority of deaths were military or civilian. Interesting that he distinguishes between ``process'' and ``outcome'' goals, with the former being goals that help the group's operations and the latter being ``their stated political ends.''  Assume that terrorism is a communication strategy that signals costs of non-compliance to a target country\\
\underline{Data \& methods} Codes 125 ``violent substate campaigns'' for ``every group ever designated'' as a Foreign Terror Organization. Dependent variable is whether the group's campaign resulted in government compliance; important independent variables are target selection, fighting capacity, stated political objective. Model is logit (model 1) and ordered logit (model 2). \\
\underline{Findings} Terror campaigns are less effective than guerilla campaigns: groups that mainly attack military targets ``regularly'' achieve policy ends, while those that mainly attack civilians tend not to gain stated political goals.  \\
\underline{Notes} The question that I think that Abrahms elides is who the audiences for terrorism are: Abrahms assumes that the audience being influenced is always the government. But, terrorism in the context of many civil wars is less about the government than about compelling civilians and populations.  Issue is also about taking stated political goals at face value. \\

\textbf{Berrebi, Claude and Esteban F. Klor. 2008. “Are Voters Sensitive to Terrorism?” American Political Science Review 102(3).}

\underline{Keywords:} empirical; 

\underline{Research question and research area} Applying disaggregated data to underlying question of how terrorism influences voting patterns. \\
\underline{Definition of terrorism:} Terror attacks included if caused at least one non-combattant Israeli death.
\underline{Summary of argument and mechanism:} Terror attacks trigger residents of locality to alter their daily routine in response to perception of insecurity, (implied negative) influence on locality's economy and expected future income. These features antagonize the residents and makes them less supportive of concessions to Palestinian authority. Attacks in a locality also make the conflict more salient to residents of the locality [Note: not clear why this wouldn't be expected to apply to neighboring localities.]\\
\underline{Data \& methods:} Difference-in-difference. (Assumes that terror attacks are the only thing that differentiates patterns in voting across localities.) Dependent variable is vote share for political parties, for 1988, 1992, 1996, 199, and 2003. Terrorism measured via data on non-combattant Israeli fatalities from terror attacks.\\
\underline{Findings} Terrorism does cause polarization of the Israeli electorate, with a terror attack in a locality adding 1.35 percentage points to the locality's support for the political right.\\
\underline{Notes on related papers} Note that this study shares the general problem of the artificiality of borders for some kinds of questions. See also the Condra and Shapiro ``who takes the blame'' article. \\

\textbf{Condra, Luke. N. and Jacob N. Shapiro. 2012. “Who Takes the Blame? The Strategic E↵ects
of Collateral Damage.” American Journal of Political Science 56 (1): 167-187.}


\underline{Keywords:}

\underline{Research question and research area}\\
\underline{Summary of argument and mechanism} Civilians are strategic agents, who will provide information on insurgents whom they are dissatisfied with, when insurgents kill civilians the civilians provide more information to the state. Note that they do not include ``intimidation killings'' in their data, which seems like an important omission when talking about civilian decisions about whether to provide information--- intimidation should drive down all reporting to Coalitions.  \\
\underline{Data \& methods} Daily trends in battlefield violence in Iraq from February 2004 through February 2009. Deaths from the Iraq Body Count, attacks from the MNF-I SIGACTS III database at the Empirical Studies of Conflict. Unit of analysis is district-week. \\
\underline{Findings} Find that insurgents and coalition forces pay penalties for violence against civilians- but unevenly throughout the country. The effect is strongest in urban environments.\\
\underline{Notes on related papers} (From Problems in IR reading)Kocher, Shapiro, Condra: focus on rational peasant model (Popkins) vs Lyall's work in Chechnya that has tended to view state forces as the active agents.\\


\textbf{Fortna, Virginia Page. 2015. “Do Terrorists Win? Rebels’ Use of Terrorism and Civil War
Outcomes.” International Organization 69(3): 519-556.}

\underline{Keywords:}

\underline{Research question and research area} Fortna's research question, ``how effective is terrorism,'' places her in the literature on effectiveness of terrorism. (Lit review on pp 521:522 has nice overview of the pro/con sides of whether terrorism is effective)\\
\underline{Implied/Stated model of terrorism:} Defines terrorist rebel groups as one that ``uses a systemic campaign of indiscriminate violence against public civilian targets to influence a wider audience'' (pg 523). Observe that this definition misses the norm-violating attribute that other thinkers, such as Crenshaw, include. This relies heavily on Jessica Stanton's typology and coding of terrorism.\\
\underline{Summary of argument and mechanism} Assumes that all agents are participating in a civil war, and have a hierarchy of preferred outcomes of the war with rebel victory $>$ negotiated settlement $>$ fizzle out $>$ government victory (worst)  \\
\underline{Data \& methods} Data is restricted to civil wars\\
\underline{Findings} Terrorism undermines effectiveness in concessions and bargaining, less bad against democracies but still never effective; wars with terrorism are longer wars thus suggesting that terrorism is effective as a tool for organizational survival.\\
\underline{Notes on related papers}\\

Terrorism week 7\\

Key for the summaries:
\underline{Keywords:}

\underline{Research question and research area}\\
\underline{Model of terrorism}\\
\underline{Summary of argument and mechanism}\\
\underline{Data \& methods}\\
\underline{Findings}\\
\underline{Notes on related papers}\\

\textbf{Bueno de Mesquita, Ethan and Eric Dickson. 2007. “The Propaganda of the Deed: Terrorism,
Counterterrorism, and Mobilization.” American Journal of Political Science 51(2).}

\underline{Keywords:} game theory, terror-state dynamics; public opinion

\underline{Research question and research area} When is terrorism an effective tactic to mobilize public opinion for an extremist group and why can groups provoke states into overreaction? Bueno de Mesquita is in the school that assumes the group-level motivation for attacks is to benefit from government counter-response that radicalizes the group's base and that terror groups and states compete over public opinon.\\
\underline{Model of terrorism} Note that the examples he cites in the motivating examples are generally large guerrilla movements (Hamas, IRA), though he did name check ETA. \\
\underline{Summary of argument and mechanism} Two causal mechanisms: 1) state response tells public whether governments are ``strong'' or ``weak'' type and 2) the economic externalities generated by armed conflict make public inclined to participate by lowering their opportunity costs. Public opinion is altered by level of economic damage inflicted by state counter-terrorism.\\.
\underline{Data \& methods} game theory, PBE solution concept. Actors are state (one of two types), extremist faction, aggrieved population \\
\underline{Findings} Most of analysis dones via comparative statics. ``Findings'' are explorations of implications of the solution spaces.\\
\underline{Notes on related papers} This paper is in a long chain of terrorism papers from EBdM that assume that economic opportunity costs influence recruitment into terrorism. Observe that there are a package of assumptions here that I'm not necessarily sure about, and in particular that terror activity supplants economically-generative activities. Nights-and-weekends operatives don't seem to exist in his world. \\

\textbf{Crenshaw, Martha. 2002. “The Logic of Terrorism: Terrorist Behavior as a Product of Strategic Choice.” in Origins of Terrorism: Psychologies, Ideologies, Theologies, States of Mind, ed. Walter
Reich. p. 54-66.}

\underline{Keywords:}

\underline{Research question and research area} Description of how terrorism can be the ``expression of a political strategy,'' argues for a way to reconcile group-level rationality with the actions of terror groups. Not theoretically cutting-edge anymore, but Crenshaw always has a great eye for vignette. \\
\underline{Model of terrorism} Terrorism is a choice of organizations that are constrained due to a lack of popular support and, in general, violence is a strategy that group use to compensate for a lack of numbers. Can also follow from technological changes that favor violent groups vs the state (dynamite, transportation, phones) or tactical diffusion. Terrorism has an  ``extremely useful agenda-setting function'' and, in an ideal case (for the group) ``inspires resistance by example'' \\
\underline{Data \& methods} n~/a\\
\underline{Findings} n~/a\\
\underline{Notes on related papers} This paper has been widely re-printed.\\

\textbf{Kydd, Andrew H. and Barbara F. Walter. 2006. “The Strategies of Terrorism,” International
Security Vol. 31, No. 1, p. 49-80.}

\underline{Keywords:}

\underline{Research question and research area} What strategies to terror groups use and when do they work?\\
\underline{Model of terrorism} Terrorism is the ``use of violence against civilians by non-state actors to attain political goals.'' Give a five-dimension summary of most common goals: regime change, territorial change, policy change, social control, status quo. Also say that information access~/control and regime type are important intervening variables. \\
\underline{Summary of argument and mechanism} Terrorism is a form of costly signaling done by groups that can't impose their will ``by force of arms'' and their violence is designed to persuade audiences to do what they want by imposing costs. ``Terrorists are forced to display publicly just how far they are willing to go to obtain their desired results.''  This happens through five strategies: attrition; intimidation; provocation; spoiling; outbidding\\
\underline{Data \& methods} Mostly a theoretical piece, but does have some summary statistics classifying US-designated organizations by their primary goal. \\
\underline{Findings} n~/a\\
\underline{Notes on related papers} Kydd and Walter propose a typology based on strategy and goals, but they don't connect the attrition and outbidding strategies don't have accompanying goals.\\

\textbf{Min, Eric. 2013. “Taking Responsibility: When and Why Terrorists Claim Attacks.” Paper presented at the 2013 meeting of the APSA and available at http://papers.ssrn.com/sol3/ papers.cfm?abstract\_id=2299920.}

\underline{Keywords:} hypothesis testing; violent political communication

\underline{Research question and research area} When do terror groups claim attacks? (Claims to test Kydd and Walter's theories of strategies.)\\
\underline{Model of terrorism} Frames terror attacks as being essentially a costly signal intended to foment uncertainty. Assumes that claiming credit is risky but that it is an activity that has strategic and communicative value\\
\underline{Summary of argument and mechanism} Groups with a concrete agenda feel that opportunity to spread their message is worth the risk of being associated with violence.\\
\underline{Data \& methods} Data on claimed vs unclaimed attacks from GTD. 27,576 observations. DV is whether the GTD lists the attack as being claimed, IDVs are array of covariates about the attack and group-- not much about the strategic environment. Model is a logit. Also tested on country-level data for Iraq, Pakistan, India, Afghanistan.\\
\underline{Findings} Probability of claim increasing (statistically significantly) in: polity, suicide bombings, GDPPC, existence of other groups, week-long proximity to another attack. Decreasing in: election years, assassination attacks, religious motivation\\
\underline{Notes on related papers} This paper is a theoretical mess, I think that it is ripe for a re-do.\\

\textbf{Plumper, Thomas and Eric Neumayer. 2014. “Terrorism and Counterterrorism: An Integrated Approach and Future Research Agenda.” International Interactions 40(4): 579–589.}

\textit{Note this is in a different style, because I already have a summary}
This is an overview piece.  P\&T provide an overview of their argument, expressed here and elsewhere, that terror organizations are strategic actors. From the strategic actions of terror groups, they conclude that domestic and international targets have differing counter-terror incentives, and particular that international  organizations. The authors conceptually divide the overarching strategy of a group from the proximate strategy of their attack: the latter is influenced by overarching goals but also a function of organizational dynamics of the group as well as the material that they have access to. 

T\&P highlight several areas that need more theoretical and analytical work:  most interesting is their description of how organizational structure influences tactical operations, and their division of terror groups into those in which leaders engage in operations (here they use the Red Army Faction as an  exemplar) and those in which leaders direct foot soldiers (such as Usama bin Laden)

There is a nice congruence here with Crenshaw’s argument that terror groups who can afford to field full-time fighters are operationally quite different from those that engage only part-time militants. The former can be more ambitious, but I wonder if they also have a harder time hiding the activities of their operatives (who would have a more difficult time melting back into the population).  Part-time fighters also imply a more geographically restricted 
scope, because they must remain available for their other employment. Also, full-time groups require more funding, which may impose additional constraints.

\underline{Week 9}

\underline{Research question and research area}\\
\underline{Summary of argument and mechanism}\\
\underline{Data \& methods}\\
\underline{Findings}\\
\underline{Notes on related papers}\\

\textbf{Crenshaw, Martha. 2007. “Explaining Suicide Terrorism: A Review Essay.” Security Studies Vol. 16 no. 1, p. 133-162.}

\underline{Keywords} review essay; 

\underline{Research question and research area} Crenshaw provides a review of the use of suicide terrorism, and what reserachers have found to be important determinants. She is primarily focused on the degree to which existing studies of suicide terrorism and  highlights the lack of methodological and definitional coherence.\\
\underline{Summary of argument and mechanism} This is a survey article, so there is no real theory being proposed.\\
\underline{Data \& methods} ``Data'' are thirteen existing articles about the use of suicide terrorism \\
\underline{Findings} Crenshaw is not impressed. However, she is insightful about gaps in research design and in identifying how articles fit together, and how their underlying assumptions compare to each other.\\
\underline{Notes on related papers}\\

\textbf{Pape, Robert A. 2003. “The Strategic Logic of Suicide Terrorism.” American Political Science Review Vol. 97, No. 3.}
Keywords: suicide terrorism, descriptive statistics

\underline{Research question and research area} Pape is intereted in why groups use suicide terrorism and argues that the tactic is a sucessful way to coerce liberal democracies to provide terroritial concessions. Pape defines suicide terrorism as being an attack carried out in such a way that the attacker does not expect to survive the mission.  As Crenshaw points out, this is a fuzzy (and vauge) definition.\\
\underline{Summary of argument and mechanism} Suicide terrorism is a ``strategy of coercion'' that is successful by inflicting enough pain on the target so as to compel the government to change.\\
\underline{Operative definition of terrorism} Pape divides terrorism into ``demonstrative'', ``destructive'' and ``suicide terrorism.'' The first is intended to generate an audience for the group, the second to coerce opponents and mobilize support, and the third (``the most aggressive form of terrorism'') values coercion over the group's constituents. Note that throughout the paper, Pape has an underlying model of suicide terrorism in which the tactic is negatively responded to by the group's supporters and that supporters view such attacks as wasteful of life. I think that this is a naieve perspective. \\
\underline{Data \& methods} Pape uses ``suicide campaigns'' as his dependent variable. He records 188 suicide terror attacks between 1980-2001, which he clusters into 16 ``campaigns'' that have been organized by terrorists into coherent strategic blocks. This is a controversial sorting. He also has a mini-case study on Hamas.\\
\underline{Findings} Using suicide terrorism is positively correlated with government concessions; sucidie terrorism is a rational strategy.\\
\underline{Notes on related papers} We savaged this paper in class, and there is a lot of dodgy content in the theory and empirics. However, it is worth noting that the heirarchy of types of terrorism, in escalating level of destructiveness, is actually a decently subtle conceptual point that many of the theorists of terrroism have had a hard time with.\\


\textbf{Ashworth, Scott, Johsua D. Clinton, Adam Meirowitz, and Kristopher W. Ramsay. 2008. “Design, Inference, and the Strategic Logic of Suicide Terrorism.” American Political Science Review Volume 102 (2): 269-273;  Pape, Robert A. 2008. “Methods and Findings in the Study of Suicide Terrorism.” American Political Science Review Volume 102 (2): 275-277;) Ashworth, Scott, Johsua D. Clinton, Adam Meirowitz, and Kristopher W. Ramsay. 2008. “Design, Inference, and the Strategic Logic of Suicide Terrorism: A Rejoinder.” Available at http://www.princeton.edu/~clinton/WorkingPapers/ACMRResponse.pdf}

Methodological spat about the selection of observations in Pape's study of suicide terrorism and particularly whether he selected for the dependent variable. If you're into that kind of thing. Jordan did these summaries for class. I'm scraping them because they're good!

Ashworth et al. (2008a)
\underline{Key point} By only looking at instances of suicide terror, Pape samples on the dependent variable.
This means that he can’t shed light on why suicide terror happens, since he doesn’t compare the cases where it occurs to the cases where it does not. While he can speak to the probability of an occupation given suicide terror, he cannot speak to the probability of suicide terror given an occupation. You also can’t “reverse engineer” this through Bayes’ Rule since Pape’s data doesn’t inform you of the probability of an occupation given no suicide terror, which you’d need the denominator. Once Ashworth et al. account for the uncertainty this leads to in Pape’s estimates, we see that Pape’s study can tell us almost nothing about the probability of suicide terror conditional on foreign occupation.

Pape's response is pretty lame: boils down to ``read my book!'' The exachange devolves from there.\\

\textbf{Pedahzur, Ami and Arie Perliger. 2006. “The Changing Nature of Suicide Attacks - A Social Network Perspective.” Social Forces 84 (4): 1987-2008.}

Keywords: social networks; terrorism\\

\underline{Research question and research area} Pedahzur and Perlinger want to know how people become radicalized into becoming suicide bombers and take a social network approach. They are one of the few in the suicide bombing literature that really push the divergence in motivation and goals between leaders and foot soldiers/fighters.\\
\underline{Summary of argument and mechanism} Arguement is that suicide bombers are recruited by diffuse networks of personal associates rather than in a top-down process.\\
\underline{Data \& methods} Map a social network of Palestinian suicide bombers from open-source information, including press releases, organization websites, and news articles. Created three levels of ties: family ties, long-term friendships, and acquantances.\\
\underline{Findings} One of their weirdest findings is that recruitment networks of sucicide bombers crosses organizations (with the same networks producing suicide bombers for more than one group.) They also find that suicide bombers are peripheral to their networks, and thus conclude that ``suicide bombers are not recruited nor do they undergo a training process.'' Unclear how they generate these conclusions from their data. \\
\underline{Notes} The connection between theory and methods are somewhat incoherent in this paper, but there are some observations about group structure in the introduction tha are worth keeping. I think that they are somewhat blase about the skills needed to create a diffuse bombing campaign--but others with experience in these things are more convinced. \\


\textbf{Wade, Sara Jackson and Dan Reiter. 2007. “Does Democracy Matter?: Regime Type and Suicide Terrorism.” Journal of Conflict Resolution Vol. 51, No. 2, 329-348.}


\end{document}
