\documentclass{article}[12pt]
\usepackage{setspace}
\usepackage{listings}
\lstset{language=R}   
\doublespacing
\usepackage[letterpaper, portrait, margin=1in]{geometry}
\setlength\parindent{0pt}
%\setlength{\parindent}{4em}
\usepackage{multirow}
\usepackage{graphicx}
\usepackage{amsmath}
\usepackage[sort,round]{natbib}
\usepackage{fancyhdr}
\pagestyle{fancy}

\begin{document}
1/16/16 Initialization
this is a document to provide 1-2 paragraph writeups of papers that I have read in classes. Short-term goal is for quals prep.
the document gets backed up to \\
https://github.com/margaretfoster/PaperSummaries, so that I can have Beeminder track my quals prep 
while I develop good habits.

Summary outlines:

Keywords provide a shorthand for broad type of paper (theory development, disaggregation, test theory, introduce data., literature review....); empirical approach (formal model, quant, computational)

Structure and keywords:

Citation

\underline{Keywords:}

\underline{Research question and research area}\\
\underline{Summary of argument and mechanism}\\
\underline{Data \& methods}\\
\underline{Findings}\\
\underline{Notes on related papers}\\

\section{International Relations}
(Paper summaries for Democratic Durability project and for traditional IR reading)


\textbf{Gunitsky, Seva. "From Shocks to Waves: Hegemonic Transitions and Democratization in the Twentieth Century." International Organization 68.03 (2014): 561-597.}

Gunitsky proposes a theory of “wave” of institutional change in state regime types in order to answer the question of why democratization of states has occurred in spurts rather than linearly. The mechanism is that that changes in the relative power of hegemons— particularly sudden changes or shocks— creates “windows” in which it is relatively less costly for a hegemon to export their own regime type, both through direct imposition and through influence and the design of international institutions. [[Observe that there is an assumption that hegemons want to export their own regime types.]]
\underline{Empirical test}, based on country-year from 1900- 2000, with national level of democracy as the dependent variable (measured through Polity IV and a Boix, Miller, and Roasto binary measure); shifts in power conducted through changes in CINC score. The supplementary materials indicate that the models he used are OLS for the version with a Polity DV and logit for the binary DV.  A followup model used fixed-effects specification for each of three hegemons.
\underline{Concerns:} Assumption that post CW is driven by US strengthening and so is a democracy-promotion period, but the cinc score that he uses actually shows the US remaining the same and China rising relative to the Soviet Union. [[Unclear what type of regime China would be exporting.]] Concern about whether OLS/Logit is appropriate modeling strategy.


\section{Social Networks / Behavioral}
(Paper summaries for 2016 Poli Sci Social Networks Class)

\textbf{Ryan, John Barry. ``Social networks as a shortcut to correct voting.'' American Journal of Political Science 55.4 (2011): 753-766. APA}\\
Ryan presents the results of an experiment about how people use information from their social network to determine which of several candidates is likely to best represent their interests. The motivation is a gap between the predictions of formal theory, which predicts that listeners should judge the credibility of messengers according to their individual merits, and the \textit{autoregressive influence theory} that proposes that speakers are influential to the degree that their messages conform to the other messages that the voter is receives (or accepts?). \underline{Implied Mechanism:} The paper tests the salience mechanism, in which messages from other actors, with potentially divergent preferences, influences how actors weigh their decisions.

The experiment is set up so that individuals decide on one of two computer-generated candidates to vote for. There are three types of voters: 1) control group has no contacts with other players while the treatment group exchanges messages; 2) partisan voters get extra benefits if their candidate wins, independent voters are indifferent; 3) informed voters get extra information about the non-partisan benefits of each candidate. The \textit{results} are consistent with the predictions of the autoregressive influence theory: uninformed voters benefited from having social information, informed subjects were harmed unless they were primarily contacted by people with the same preferences that they had. People tended to vote as their social networks voted. The magnitude of the effects seems very small.

\textbf{Leighley, Jan E. 1990. “Social Interaction and Contextual Influences On Political Participation.” American Politics Research 18(4): 459–475.}\\

\underline{Keywords:} social networks; voting; empirical test of theory

\underline{Research question and research area:} Leighley tests then-current theories about how social networks affect political participation. She network size, politicization, and within-network political conflict as predictors of four indicator variables: voting, campaigning, contacting, and 
\underline{Summary of argument and mechanism:} Leighley is carrying out an empirical test of arguments that social networks influence activity. Both she and the theories that she cites seem to have a strong assumption of community homophily.
\underline{Data \& methods:} Data from the 1976 American National Election Study (ANES) survey, in which respondents were asked to list the people with whom they discussed the ``most important national problem.''  I think that the social network IDVs are brought in via a count of discussants and indicator variables for conflict and politicization. Empirical test is a logit model of 724 individuals.
\underline{Findings};
\underline{Notes on related papers or followup:} The conflict measure would be interesting to look at in a multiparty system, with more fluid voting patterns. Would also be interesting to look at network-based diffusion of strategic voting. 

\textbf{McAdam, Doug, and Ronnelle Paulsen. 1993. “Specifying the Relationship between Social Ties and Activism.” American Journal Of Sociology 99(3): 640–67.}

\underline{Keywords:} networks; theory-testing

\underline{Research question and area:} McAdam and Paulsen connect to the literature of mobilization into high-risk political organizations. They primarily connect with the sociology literature, and, for example, don't make reference to free-rider problems that preoccupy the civil wars perspective on this question.\\
\underline{Summary of argument and mechanism:} McAdam and Paulsen are testing theories about how interpersonal ties and membership in organizations influence identity-based mobilization. The theory holds that social networks reinforce identity-based recruitment appeals by affirming that participation would support salient identity.
\underline{Data \& methods} Social network data generated from 1068 applications for participation in the 1964 Mississippi Freedom Summer project, coded networks based on surveys from participants, rejections, and no-shows and 556 followup surveys. Network-related independent variables operationalized as binary variables coding presence of listed organization or source of support.  Dependent variable was binary for participation in freedom summer; logit (?) statistical model.
\underline{Findings:} Personal ties are most important as sources of influence (vs conduits for information); suggestion that best networks for mobilization is weak ties between lots of tightly linked local organizations. Organizational support more important than individual ties for encouraging participation in a collective action.
\underline{Notes on related papers:} Lots of applicability here to mobilization for violent extremist organizations. 

\textbf{Mutz, Diana C. 2002. “The Consequences of Cross-Cutting Networks for Political Participation.” American Journal of Political Science 46(4): 838–55.}

\underline{Keywords:} networks; theory testing

\underline{Research question and research area:} Does homogeneity of political belief in networks influence political participation?
\underline{Summary of argument and mechanism:} Non-homogeneous networks introduce uncertainty by transmitting contradictory information thus making people take longer to decide who to vote for; non-homogenous networks depress willingness to publicly share a position because can not make everyone in the network happy.
\underline{Data \& methods:} Used two 1996 surveys about political participation and political interaction within social networks. Logit model with network information included as binary and count variables.
\underline{Findings:} Results most strongly suggest that cross-cutting networks depress willingness to be demonstrative, rather than that the influence makes people confused about what they want.
\underline{Notes on related papers:} I feel like the conflict-avoidance mechanism here would be a natural fit to update to the 21st century via questionnaires about people's activities on social networking sites. (\textit{e.g.}: in the past x months, have you avoided sharing content that promotes a political or social view because of concern about how people will respond?)

\textbf{McClurg, Scott D. 2003. “Social Networks and Political Participation: The Role of Social Interaction in Explaining Political Participation.” Political Research Quarterly 56(4): 449–64.}

\underline{Keywords:}

\underline{Research question and research area:} Seems to speak most closely to the social capital literature and social determinants of political participation. 
\underline{Summary of argument and mechanism:} Mechanism is explicitly about information-sharing functions of social networks.
\underline{Data \& methods:} Survey data from 1984 Presidential election gathered in South Bend, Indiana. Unit of analysis is the discussion dyad. Dependent variable is a composite count of political activities beyond voting. Model is negative binomial regression.
\underline{Findings:} 
\underline{Notes on related papers:}
combining this article and the  Mutz findings and adding to the context  of social networks is interesting:
then we have a  clear vector of information (both  passive via FB  events) and active,  via people sharing  their views and  things like articles,  BUT also because some online social networks tend to be maximalist (especially Facebook) they also foster the  development of the type of cross- cutting networks that Mutz finds tends to make people cagy about political activism. I bet that beyond presidential election, social campaigns like support for Black Lives Matter would be a good test case of information sharing through networks (having partisans in online networks might increase access to POV and stories 

\section{Terrorism}

\textbf{de Mesquita, Ethan Bueno. "The political economy of terrorism: A selective overview of recent work." The Political Economist 10.1 (2008): 1-12.
APA} (Week one)

Key: literature review.
 
This is a literature review essay about the then-current strand of the literature of academic study of terrorism, with a particular focus on the outcomes of formal models.  There is not a theory that is being tested, so there isn't data to assess. Note that, as with all EBdM papers about terrorism, he presents the results of his own research as if the conclusions are settled questions in the field.

\textbf{Aksoy, Deniz and David B. Carter. 2014. “Electoral Institutions and the Emergence of Terrorist Groups.” British Journal of Political Science 44(1): 181-204.}

Key: disaggregation; quant.

Research question and area: Aksoy and Carter build on the literature about how domestic institutions-- and particularly democratic institutions-- influence the emergence of terror groups. They are in the `` do democracies experience more terrorism'' research agenda. A\&C look at how different types of terror groups (``within-system'' who have goals that can be fulfilled within existing system and ``anti-system'' whose goals require complete overthrow of the existing political system. Irredentist and separatist groups are called anti-system.) are differently influenced by democratic institutions, and argue that previous research hasn't found ``robust'' evidence of a connection between electoral rules and terrorism because it combines all groups together. Thus, this is a \underline{disaggregation} type paper.
Articulated/assumed definition of ``terrorism'': Terrorism is a political organization trying to ``achieve their goals by using violence against civilians''
Summary of argument and mechanism: Democratic institutions should reduce operation of within-system groups because they provide a way for the goals of the organization to be met; should not have an impact on anti-system groups.
Data \& methods: Look only at domestic groups; data is from Jones and Libick (2006), which has 648 terror grops from 1968-2006, and TWEED, which has 286 groups from 150-2004. Three main DVs: emergence of any group, emergence of anti-system group, emergence of within-system group. Unit of analysis is country-year. Primary IDVs: indicator variable for democracy, domestic institutions from IAEP and Democratic Electoral Systems Around the World Project, log of district magnitude, and various controls. Model is logit.
Findings: permissive electoral systems associated with reduced risk of within-system group emergence but have no impact on anti-system groups. 
Notes on related papers:

\textbf{Blomberg, S. Brock, Gregory D. Hess and Akila Weerapana. 2004. “Economic Conditions and Terrorism.” European Journal of Political Economy 20(2):463-478.}

\underline{Keywords:} quant; computational; theory testing.

\underline{Research question and area:} Blomberg et al. contribute to the literature on how economic conditions influence terrorist activities.\\
Terrorism definition: differentiate between ``rebellion'' which happens when a group that wants to change the status quo overthrows the government and takes power and ``terrorist attacks'' when dissident groups use ``terror activities'' to ``increase voice in economy'' but are not able to overthrow government. Note the almost tautological definition and the build-in assumption that terror attacks are intended to express a perspective on economy. Terror attacks conceptualized as ``low-intensity attacks'' designed to signal discontent with status quo. [This definition would include the far-left / far-right vandalism.]
\underline{Summary of argument and mechanism:} Groups choose between rebellion and terror attacks according to whether the country can give in to the dissidents. Strong institutions attracts terrorism; weak ones attract rebellion. [Observe that this outcome is built into their definition: weaker states should disproportionately attract attacks that threaten the state institutions because the institutions are more vulnerable.] They have a strong underlying assumption that attacks are intended to signal discontent. Assumption that attacks are either signals of anger or earnest attempts to overthrow government.
\underline{Data \& methods:} Dataset of 130 countries from 1968-1991, country-year unit of analysis. Data from ITERATE, which is transnational attacks and thus doesn't really match their theory, with economic indicators. Build a model of economic state change and terrorism/no terrorism through Markov processes. Do not appear to have any measure of the rebellion actions of their theory and look at conditional probabilities of state change. [Methodological question: are Markov chains at all appropriate for strategic interactions? In general, they are only applicable for linear systems.]
\underline{Findings:} In democratic, high income, countries, recessions increase probabilities of terror attacks. 
Notes on related papers:


\textbf{Chenoweth, Erica. 2013. “Terrorism and Democracy.” Annual Review of Political Science 16: 355–378.}

\underline{Keywords:} literature review.
\underline{Research question and research area:} Chenoweth provides an overview of research on the question f whether democracies disproportionately attract terrorist violence. \\
\underline{Operational definition of terrorism:} Here Chenoweth works from the basis that terrorsim is the deliberate use or threat of force against noncombatants by a non-state actor in pursuit of a political goals.
\underline{Summary of argument and mechanism:} Surveys underlying arguments about why democracies might, or might not, attract terrorism that are found in the articles that she synthesizes. 
\underline{Data \& methods:} Varies based on paper surveyed, but in general the studies all based on observed attacks and the GTD features heavily in both the motivating introduction and in the studies that she discusses. 
\underline{Findings:} She synthesizes existing studies to argue that terrorism is not a strategy of desperation, in particular that there is a non-linear relationship, with more violence in anocracies and partial democracies. As well, the patterns of the data change after 2001, with terrorism shifting from democracies to partial democracies and authoritarian regimes
\underline{Notes on related papers:} As a general note- this is probably a major source, and likely even a template, for any question about democracies and terrorism.

\textbf{Drakos, Kostas and Andreas Gofas. 2006. “The Devil You Know but are Afraid to Face:
Underreporting Bias and its Distorting Effects on the Study of Terrorism.” Journal of Conflict Resolution 50(5):714-735}


\underline{Keywords:} Terrorism; methodology

\underline{Research question and research area:} This is primarily a methodological paper- Drakos and Gofas are concerned with underreporting bias over terror attacks, leading to the situation in which the count of attacks that we see in news is lower than full sample of terrorist attacks and systematically underrepresents attacks in autocracies without a free press. This paper also speaks to the democracy and terrorism literature, which is where they primarily orient their literature review.
\underline{Construction of terrorism:} Terrorism is violence, or threat of violence calculated to create an atmosphere of fear and alarm. [Observe that this allows for states to perpetuate terrorism, which might be particularly systematically underreported.] 
\underline{Summary of argument and mechanism:} Argue that bias depends on degree of press freedom in country, so less freedom means underpresentation in aggregate databases like the GTD and findings that democracies attract more terrorism. Argue that cannot just put in a control for press freedom in part because it is a component of Polity.
\underline{Data \& methods:} Use concept of ``thinning'' in which count-data is generated as the true number of events weighted by some country-specific probability of inclusion. Assume that policy is ``fundamental determinant'' of underreporting bias because of inclusion of press freedom. Data sample is country-year unit for 153 countries from 1985-1998. Data from MIPT and Rand. Determine underreporting by looking at differences between two linear models (pg 732 of article).
\underline{Findings:} Find evidence that they claim is strongly indicative of underreporting bias in many countries, but they avoid making country-specific observations.
\underline{Notes on related papers:} Wonder if one could use Bayesian techniques to extend this approach, for example, by saying that the posterior would be underreporting $|$ observed attacks. 





\end{document}
