\documentclass{article}[12pt]
\usepackage{setspace}
\usepackage{listings}
\lstset{language=R}   
\doublespacing
\usepackage[letterpaper, portrait, margin=1in]{geometry}
\setlength\parindent{0pt}
%\setlength{\parindent}{4em}
\usepackage{multirow}
\usepackage{graphicx}
\usepackage{amsmath}
\usepackage[sort,round]{natbib}
\usepackage{fancyhdr}
\pagestyle{fancy}

\begin{document}
1/16/16 Initialization
this is a document to provide 1-2 paragraph writeups of papers that I have read in classes. Short-term goal is for quals prep.
the document gets backed up to https://github.com/margaretfoster/PaperSummaries, so that I can have Beeminder track my quals prep 
while I develop good habits.

Summary outlines:

Structure:
Citation
Research question and research area
Summary of argument and mechanism
Data \& methods
Findings
Notes on related papers

\section{International Relations}
(Paper summaries for Democratic Durability project and for traditional IR reading)


\textbf{Gunitsky, Seva. "From Shocks to Waves: Hegemonic Transitions and Democratization in the Twentieth Century." International Organization 68.03 (2014): 561-597.}

Gunitsky proposes a theory of “wave” of institutional change in state regime types in order to answer the question of why democratization of states has occurred in spurts rather than linearly. The mechanism is that that changes in the relative power of hegemons— particularly sudden changes or shocks— creates “windows” in which it is relatively less costly for a hegemon to export their own regime type, both through direct imposition and through influence and the design of international institutions. [[Observe that there is an assumption that hegemons want to export their own regime types.]]
\underline{Empirical test}, based on country-year from 1900- 2000, with national level of democracy as the dependent variable (measured through Polity IV and a Boix, Miller, and Roasto binary measure); shifts in power conducted through changes in CINC score. The supplementary materials indicate that the models he used are OLS for the version with a Polity DV and logit for the binary DV.  A followup model used fixed-effects specification for each of three hegemons.
\underline{Concerns:} Assumption that post CW is driven by US strengthening and so is a democracy-promotion period, but the cinc score that he uses actually shows the US remaining the same and China rising relative to the Soviet Union. [[Unclear what type of regime China would be exporting.]] Concern about whether OLS/Logit is appropriate modeling strategy.


\section{Social Networks / Behavioral}
(Paper summaries for 2016 Poli Sci Social Networks Class)




\end{document}
