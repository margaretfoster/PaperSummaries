\documentclass{article}[12pt]
\usepackage{setspace}
\usepackage{listings}
\lstset{language=R}   
\doublespacing
\usepackage[letterpaper, portrait, margin=1in]{geometry}
\setlength\parindent{0pt}
%\setlength{\parindent}{4em}
\usepackage{multirow}
\usepackage{graphicx}
\usepackage{amsmath}
\usepackage[sort,round]{natbib}
\usepackage{fancyhdr}
\pagestyle{fancy}

\begin{document}
1/16/16 Initialization
this is a document to provide 1-2 paragraph writeups of papers that I have read in classes. Short-term goal is for quals prep.
the document gets backed up to https://github.com/margaretfoster/PaperSummaries, so that I can have Beeminder track my quals prep 
while I develop good habits.

Summary outlines:

Keywords provide a shorthand for broad type of paper (theory development, disaggregation, test theory, introduce data., literature review....); empirical approach (formal model, quant, computational)
Structure:
Citation
Research question and research area
Summary of argument and mechanism
Data \& methods
Findings
Notes on related papers

\section{International Relations}
(Paper summaries for Democratic Durability project and for traditional IR reading)


\textbf{Gunitsky, Seva. "From Shocks to Waves: Hegemonic Transitions and Democratization in the Twentieth Century." International Organization 68.03 (2014): 561-597.}

Gunitsky proposes a theory of “wave” of institutional change in state regime types in order to answer the question of why democratization of states has occurred in spurts rather than linearly. The mechanism is that that changes in the relative power of hegemons— particularly sudden changes or shocks— creates “windows” in which it is relatively less costly for a hegemon to export their own regime type, both through direct imposition and through influence and the design of international institutions. [[Observe that there is an assumption that hegemons want to export their own regime types.]]
\underline{Empirical test}, based on country-year from 1900- 2000, with national level of democracy as the dependent variable (measured through Polity IV and a Boix, Miller, and Roasto binary measure); shifts in power conducted through changes in CINC score. The supplementary materials indicate that the models he used are OLS for the version with a Polity DV and logit for the binary DV.  A followup model used fixed-effects specification for each of three hegemons.
\underline{Concerns:} Assumption that post CW is driven by US strengthening and so is a democracy-promotion period, but the cinc score that he uses actually shows the US remaining the same and China rising relative to the Soviet Union. [[Unclear what type of regime China would be exporting.]] Concern about whether OLS/Logit is appropriate modeling strategy.


\section{Social Networks / Behavioral}
(Paper summaries for 2016 Poli Sci Social Networks Class)

\textbf{Ryan, John Barry. ``Social networks as a shortcut to correct voting.'' American Journal of Political Science 55.4 (2011): 753-766. APA}\\
Ryan presents the results of an experiment about how people use information from their social network to determine which of several candidates is likely to best represent their interests. The motivation is a gap between the predictions of formal theory, which predicts that listeners should judge the credibility of messengers according to their individual merits, and the \textit{autoregressive influence theory} that proposes that speakers are influential to the degree that their messages conform to the other messages that the voter is receives (or accepts?). \underline{Implied Mechanism:} The paper tests the salience mechanism, in which messages from other actors, with potentially divergent preferences, influences how actors weigh their decisions.

The experiment is set up so that individuals decide on one of two computer-generated candidates to vote for. There are three types of voters: 1) control group has no contacts with other players while the treatment group exchanges messages; 2) partisan voters get extra benefits if their candidate wins, independent voters are indifferent; 3) informed voters get extra information about the non-partisan benefits of each candidate. The \textit{results} are consistent with the predictions of the autoregressive influence theory: uninformed voters benefited from having social information, informed subjects were harmed unless they were primarily contacted by people with the same preferences that they had. People tended to vote as their social networks voted. The magnitude of the effects seems very small.


\section{Terrorism}

\textbf{de Mesquita, Ethan Bueno. "The political economy of terrorism: A selective overview of recent work." The Political Economist 10.1 (2008): 1-12.
APA} (Week one)

Key: literature review.
 
This is a literature review essay about the then-current strand of the literature of academic study of terrorism, with a particular focus on the outcomes of formal models.  There is not a theory that is being tested, so there isn't data to assess. Note that, as with all EBdM papers about terrorism, he presents the results of his own research as if the conclusions are settled questions in the field.

\textbf{Aksoy, Deniz and David B. Carter. 2014. “Electoral Institutions and the Emergence of Terrorist Groups.” British Journal of Political Science 44(1): 181-204.}

Key: disaggregation; quant.

Research question and area: Aksoy and Carter build on the literature about how domestic institutions-- and particularly democratic institutions-- influence the emergence of terror groups. They are in the `` do democracies experience more terrorism'' research agenda. A\&C look at how different types of terror groups (``within-system'' who have goals that can be fulfilled within existing system and ``anti-system'' whose goals require complete overthrow of the existing political system. Irredentist and separatist groups are called anti-system.) are differently influenced by democratic institutions, and argue that previous research hasn't found ``robust'' evidence of a connection between electoral rules and terrorism because it combines all groups together. Thus, this is a \underline{disaggregation} type paper.
Articulated/assumed definition of ``terrorism'': Terrorism is a political organization trying to ``achieve their goals by using violence against civilians''
Summary of argument and mechanism: Democratic institutions should reduce operation of within-system groups because they provide a way for the goals of the organization to be met; should not have an impact on anti-system groups.
Data \& methods: Look only at domestic groups; data is from Jones and Libick (2006), which has 648 terror grops from 1968-2006, and TWEED, which has 286 groups from 150-2004. Three main DVs: emergence of any group, emergence of anti-system group, emergence of within-system group. Unit of analysis is country-year. Primary IDVs: indicator variable for democracy, domestic institutions from IAEP and Democratic Electoral Systems Around the World Project, log of district magnitude, and various controls. Model is logit.
Findings: permissive electoral systems associated with reduced risk of within-system group emergence but have no impact on anti-system groups. 
Notes on related papers:




\end{document}
